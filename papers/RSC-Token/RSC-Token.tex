%%%%%%%%%%%%%%%%%%%%%%% file typeinst.tex %%%%%%%%%%%%%%%%%%%%%%%%%
%
% This is the LaTeX source for the instructions to authors using
% the LaTeX document class 'llncs.cls' for contributions to
% the Lecture Notes in Computer Sciences series.
% http://www.springer.com/lncs       Springer Heidelberg 2006/05/04
%
% It may be used as a template for your own input - copy it
% to a new file with a new name and use it as the basis
% for your article.
%
% NB: the document class 'llncs' has its own and detailed documentation, see
% ftp://ftp.springer.de/data/pubftp/pub/tex/latex/llncs/latex2e/llncsdoc.pdf
%
%%%%%%%%%%%%%%%%%%%%%%%%%%%%%%%%%%%%%%%%%%%%%%%%%%%%%%%%%%%%%%%%%%%

\documentclass[runningheads,a4paper]{llncs}

\usepackage[utf8]{inputenc}

\usepackage{natbib}
\usepackage{hyperref}
\usepackage{upgreek}
\bibliographystyle{apalike-fr}

\usepackage{amssymb}
\usepackage{amsmath}
\setcounter{tocdepth}{2}
\usepackage{graphicx}

\usepackage[english]{babel} % Pour adopter les règles de typographie française
\usepackage[T1]{fontenc} % Pour que les lettres accentuées soient reconnues

\usepackage{url}
\newcommand{\keywords}[1]{\par\addvspace\baselineskip
\noindent\keywordname\enspace\ignorespaces#1}
\newcommand{\ETH}{$\Upxi$}
\begin{document}

\mainmatter 

\title{RSC - Etherisc Token}

\titlerunning{RSC : Etherisc Token}

\author{etherisc hackathon team: Jake Brukhman, Stephan Karpischek, Christoph Mussenbrock}

\institute{etherisc GmbH}

\authorrunning{etherisc hackathon team}

\toctitle{Abstract}
\tocauthor{{}}

\maketitle

\begin{abstract}
The rationale behind this document is to collect the ideas and introduce a consistent wording throughout the next weeks.
\end{abstract}

\medskip

\begingroup
\let\clearpage\relax
\tableofcontents
\addcontentsline{toc}{section}{Introduction}
\endgroup

\medskip
\medskip

\section*{Introduction}

We try to formulate an economical context for an ethereum ERC-20-token which serves as a base for a 
P2P-Insurance. 

The Model should be self-stabilizing - it should run forever without (to much) 
human interference and governance. Instead, we like to employ Market mechanisms for
central governance topics like token issuing, pricing and so on.

\section{Objectives}
Before we start with describing a solution, we should write down what we want to achieve.
The following objectives should be met and any proposed solution should be matched against these.
\begin{enumerate}
    \item \textbf{Simple:} The structure of the system should be simple and auditable.\\
    \textit{Rationale:} Blockchain is still in the early days, we should start with something simple and
    intelligible. People have difficulties understanding blockchain at all, we don't want to burden them 
    with a complicated system. Acceptance and market penetration will be higher with a simple system.
    \item \textbf{Self-regulating, self-enforcing:} The contract should 
    react automatically on events and regulate accordingly.\\
    \textit{Rationale:} Adhering to blockchain principles and decentralization.
    \item \textbf{Secure:} There should be reasonable protection of customers and investors.\\
    \textit{Rationale:} Without investors, no insurance. Therefore the interests of customers and investors
    have to be protected.
    \item \textbf{Sustainable:} The contract should be able to overcome bad market conditions. \\
    \textit{Rationale:} If the contract fails for any reason, this will have negative impact on the ecosystem, 
    which should be avoided.
    \item \textbf{Extensible:} The contract should be able to growth organically according to market demands.\\
    \textit{Rationale:} If the concept is successful, hard limits will hurt.
\end{enumerate}


\section{Glossary}

We make the following assumptions for simplicity:

\begin{enumerate}
    \item We have uniform policies with a probability of default ($PD$) of 1\%.
    \item The insured value per policy ($IV$) is 100 ETH (that’s what you get in case of a damage).
    \item The policies are uncorrelated. 
    \item Damages $D$ are always total, there is no partial payout. 
    \item The expected loss ($EL$) is therefore 1\%; that’s the base premium ($BP$): 1 ETH per policy.
    \item On top of the base premium, we have some operational costs ($OC$). 
    To make things simple, we assume the operational costs to be 10\% of the base premium.
    \item Furthermore, we have to cover the interest rate ($IR$) and risk premium ($RP$) for the investors. 
    The idea is that we pay a risk-free interest rate and on top an adequate bonus for the risk. \\
    QUESTION Q1: Out of these parameters, can we calculate an adequate interest rate and risk premium for the investors? \\
    QUESTION Q2: Is there a “risk-free” interest rate ($RFIR$) in the crypto world which could serve as a baseline? \\
    \texttt{\href{https://en.wikipedia.org/wiki/Risk-free_interest_rate}
    {https://en.wikipedia.org/wiki/Risk-free\_interest\_rate}}\\
    Proposal: the ``risk free interest rate'' is the expected revenue which a rational player
    would expect for locking an amount of money for a certain time, covering the volatility risks of the crypto coin.
    \item The gross premium ($GP$) is calculated as follows:
    \begin{equation} \label{eq:GP1} GP = (IV \cdot PD) \cdot (1 + OC + IR + RP) \end{equation}
    \item Out ouf these parameters, we calculate a “Coverage ratio” ($CR$) (is this the right term?): 
    The coverage ratio is the amount of insured value I can underwrite given a certain amount of risk capital.
    For example, if the Coverage ratio is 20\%, I can insure 1,000,000 \ETH\ with a risk capital of 200,000 \ETH.\\
    QUESTION Q3: Can we calculate this coverage ratio, and how?
\end{enumerate}

\section{RSC Token}

\begin{enumerate}
    \item We create an ERC-20 Token ``RSC'', at the beginning with a cap of $TC$ Tokens.
    \item The Token is sold during an ICO at a rate of 1:1 (1 RSC = 1 \ETH).
    \item The proceeds $PR$ from selling $TS \le TC$ tokens in the ICO form the liquidity pool $LP$; $PR = LP = TS\ \Upxi$.
    \item From the liquidity pool, a ``free risk volume'' $FRV_{start}$ is generated by
    \begin{equation} \label{eq:FRV_start} FRV_{start} = \frac{LP}{CR} \end{equation}
    \item On the other hand, if we sell policies , we need to bind the risk capital to these policies.
    We denote the amount of bound risk capital with $BRV$. 
    Every sold policy reduces the $FRV$ by its $IV$ and increases the $BRV$ by the same amount.
    We designate the number of sold policies with $NP_{sold}$ and the number of active policies with $NP_{act}$. 
    After that, the remaining free risk volume is
    \begin{equation} \label{eq:FRV_rem} FRV_{rem} = \frac{LP}{CR} - NP_{act} \cdot IV \end{equation}
    \item Every expired policy increases the $FRV$ by its $IV$ and reduces the $BRV$ by the same amount. 
    The number of expired policies is denoted by $NP_{exp}$.
    \item We get the following invariant: 
    \begin{equation} \label{eq:inv} \frac{LP}{CR} = {FRV} + {BRV} \end{equation}
    \item Every damage $D$ decreases the liquidity pool $LP$ and the bound risk volume $BRV$ $D$. 
    If we assume that \eqref{eq:inv} holds, then we get
    \begin{equation} \label{eq:delta_FRV} \frac{LP - D}{CR} = ({FRV} + \Delta {FRV})  + ({BRV} - {D}) \end{equation}    
    and therefore
    \begin{equation} \label{eq:delta_FRV2} \Delta {FRV} = {D} \frac{CR-1}{CR} \end{equation}
    If $CR$ is fairly small (e.g. 10\%), a damage decreases the free risk volume $FRV$  by a factor of 9, because
    liquidity leverages the $FRV$ 
    \item If $ FRV = 0$, no more policies can be sold. 
    \item We define a lower bound $B_1$ for the free risk volume. If $ FRV < B $, 
    new tokens are minted and sold at market price, yielding proceeds of $PR_{new}$. 
    The proceeds increase the liquidity, and because of the leverage effect of the $CR$ this leads 
    to an over-proprortional increase of the $FRV$. 
    If $FRV < B$ and we want to increase $FRV$ by $\Delta FRV$ to get over the lower bound $B$ again, we solve
    \begin{equation} \label{eq:growth} \frac{LP + PR_{new}}{CR} = FRV + \Delta FRV + BRV  \end{equation} yielding
    \begin{equation} \label{eq:delta_PR} {PR_{new}} = \Delta FRV\cdot CR \end{equation}
    Again, a small increase in liquidity yields a large increase in free risk capital.
    This process can be automated in the contract: if there is need for fresh capital, the contract can 
    automatically mint new tokens and ensure that there is always enough free risk volume $FRV$ so that 
    new policies can be sold.
    \item In theory, the reverse process is also possible: If there is sustainable low demand for policies,
    and therefore the ratio of free to bound volume very high, the contract could buy tokens back and burn them,
    reducing liquidity $LP$ and accordingly the free risk volume $FRV$. 
\end{enumerate}
\subsubsection{Conclusion:} the concepts of free and bound risk volume together with liquidity and coverage rate
provide a simple and effective framework for automatic adjustment of token production. 

\section{Collecting premiums}
\begin{enumerate}
\item As we have seen, the gross premium $GP$ has some components: part of the premiums $IV \cdot PD$ will cover the 
expected damages, operational costs $OC$ have to be covered and the rest is for investors: paying a fair 
interest rate $IR$ and risk premium $RP$.
\begin{equation} \label{eq:GP2}
    GP = (IV \cdot PD) \cdot (1 + OC + IR + RP)
\end{equation}
$OC$, $IR$ and $RP$ will be paid directly and have no significant influence on the liquidity pool $LP$. 
\item The base premium $IV \cdot PD$ will cover normal damages. Depending on the risk structure and the individual
damage events, there will be periods of time where $IV \cdot PD$ will cover less than the total damages, in other
periods there will be excess.
\item In our model, we will assume that excess premiums aren't paid back to customers, but instead increase the 
liquidity pool $LP$. Reciprocally, the liquidity pool will cover excessive damages. In the next section, we 
discuss this more in depth.
\end{enumerate}

\section{Covering damages}
\begin{enumerate}
\item What happens if the base premium $IV \cdot PD$ does not cover all damages in a certain intervall of time?
After consuming $IV \cdot PD$, these excessive damages are covered by the liquidity pool $LP$.
Due to $CR \ll 1$ we can expect that in theory there are black swan events where the sum of all damages 
exceeds even $LP$. We assume that in such an event, a big number $ND$ of damages occur with 
$ND \cdot IV \gg (IV \cdot PD) + LP$. 
\item A first approach would be to pay out the whole $LP$, 
so every customer gets a partial payout
\begin{equation} \label{eq:dam1}
    PP = \frac{(IV \cdot PD) + LP}{ND}
\end{equation}
\item If $ND < NP_{act}$, a number of customers is left without any protection.
To avoid this, we should pay out no more than
\begin{equation} \label{eq:dam2}
    PL = \frac{(IV \cdot PD) + LP}{NP_{act}}
\end{equation} 
So the remaining customers have at least the same coverage as the customers which are getting their payout.
\item The same applies if $ND \cdot IV < LP$. In this case, all claims could be fully paid out, but again discriminating
the remaining customers. We end up in paying at most $\min(PL, IV)$. 
\item The payout limiting term $PL$ seems to play an important role. 
It is an upper bound for even a single payout in case of a damage. 
The contract should always be ``far enough away'' from this limit. 
\item We can introduce a second bound $B_2 < 1$ with the same functionality as the bound $B_1$ above:
if $IV > B_2 \cdot PL$, new tokens are minted until the bound is met again.
\item A third bound $B_3 > 1 $ serves as an absolute limit for accepting new policies. As soon as $IV > B_3 \cdot PL$, every customer gets only a fraction $\frac{IV}{B_3}$ of their actual claim. Therefore, $B_3$ should be not to high, but maybe e.g. $B_3 = 1.5$ would be acceptable.

\end{enumerate}

\section{Contract pseudocode}
This is a first draft of some contract, I've left out all the ERC-20 standard stuff (transfering tokens, etc.)


\bibliography{references}
\nocite{*} 

\end{document}
